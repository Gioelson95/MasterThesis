\chapter{Abstract}
\label{Abstract}
\thispagestyle{empty}

\indent One of the most attractive functions of music is that it can convey emotion and modulate a listener's mood \cite{feng2003popular}. Music can bring to tears, console us when we are grieving and drive us to love.
\\
Music information behavior studies have identified emotion as an important criterion used by people in music searching and organization.Hence, it becomes more and more significant the role of \textit{music emotion recognition}.
\\ \indent

Nowadays, to retrieve and organize user music is becoming more and more important, due to the increasing platforms of streaming, which gives the access to a catalog of billions of songs.
\\
The automatization of the perceived emotion recognition in music allows users to organize and to research in a content-centric fashion.
\\ \indent
Purpose of this thesis is to find a link between music and emotions during the listening of a song by combining audio and physiological signals analysis.
\\ \indent
Inclusion of emotions is found to be an hard task, due to the subjective nature of emotion perception. There are problems in the reliability of ground truth data and evaluation of prediction results, all issues that are not present in other data-driven tasks, like for example face recognition or speech recognition.