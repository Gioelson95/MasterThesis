\chapter{Abstract}
\label{Abstract}
\thispagestyle{empty}

\indent One of the most attractive functions of music is that it can convey emotion and modulate a listener's mood \cite{feng2003popular}. Music can bring to tears, console us when we are grieving and drive us to love.
\\
Most important thing is that music information behavior studies have identified emotion as an important criterion used by people in music searching and organization. Now become important the field of music emotion recognition.
\\ \indent
Nowdays, is more and more important to retrieve and organize users music, due to the increasing platforms of streaming, which gives the access to a catalog of billions of songs.
\\
The automatization of the recognition of perceived emotion in music allows users to organize and research music in a content-centric fashion.
\\ \indent
The purpose of this thesis is to find a link between emotion perceived and felt during the playing of a song using  analysis of the audio signal and the incorporation of physiological signals.
\\ \indent
The inclusion of emotions is an hard task, because of the subjective nature of emotion perception. There are problems in the reliability of ground truth data and evaluation of prediction results, which are not troubles in problems as face recognition or speech recognition.