\chapter{Sommario}
\label{Sommario}

\thispagestyle{empty}

\indent Una delle funzioni più attrattive della musica è che questa può trasmettere e comunicare emozioni e modulare l'umore di una persona, come descritto in \cite{feng2003popular}. La musica può provocarci lacrime, consolarci quando siamo tristi, farci innamorare.
\\
Gli studi fatti finora sulla musica, affermano che le emozioni sono un criterio importante per la ricerca e l'organizzazione dei brani musicali. Qui diventa fondamentale l'importanza del campo chiamato \textit{music emotion recognition}.
\\ \indent
Al giorno d'oggi, diventa sempre più importante il fatto di catalogare e organizzare la musica degli utenti, a causa dell'incremento di piattaforme di streaming musicale, le quali danno accesso ad un numero infinito di brani.
\\
L'automatizzazione del riconoscimento delle emozioni percepite in musica, permette all'utente di organizzare e ricercare la musica in una visione più incentrata sul contenuto.
\\ \indent
Lo scopo di questa tesi è quello di trovare il link tra emozioni percepite durante l'ascolto di un brano musicale attraverso l'analisi del segnale audio in primis, ma anche con l'utilizzo di segnali psicologici.
\\ \indent
L'utilizzo delle emozioni , in generale, è un compito difficile, a causa della natura intrinseca delle emozioni percepite. Ci sono problemi di affidabilità dei dati empirici e la valutazione del modello di predizione, che d'altra parte non sono dei problemi nei casi ben noti di \textit{face recognition} e \textit{speech recognition}.