\chapter{Introduction}
\label{Introduction}
\pagestyle{plain}

\vspace{0.5cm}

\section{Motivation}
Music has an important role in human life. More important, is that music in capable to evoke different emotions for people, but how is structured the relationship between music and emotion? We don't know yet. It's a hard problem, which have very different fields of background, from computer science, machine learning and psychology.
\\ \indent
Emotion-aware Music Information Retrieval has been difficult due to the subjectivity and temporal of emotion responses to music. The role of physiological signals related to emotions could potentially be exploited in emotion-aware music discovery.
\\ \indent
Music is the vehicle for emotions, feelings, passion and actions. With the music the composer create a narration which is purely emotional.
\\ \\
Can we measure emotions related to music?

\section{Outline of the thesis}
This thesis is organized as follows:
\\ \indent
After a brief introduction about the objective of the thesis, in chapter 2 and 3 is presented a complete overview about the main arguments in chapter 2, as Music Information Retrieval (MIR) and Music Emotion Recognition (MER), Electrodermal Activity (EDA) and other physiological data using on-body sensors.
\\ \indent
Chapter 4 is devoted to a complete overview of the state of the art about the main aspects related to chapters 2 and 3 of this thesis, in order to have a general idea about what has been done in the past and which results they have achieved.
\\ \indent
In chapter 5 is presented how the dataset we have considered is structured and what results they have reached. Is also shown our implementation of the problem.
\\ \indent
Chapter 6 is about the results we have achieved and the comparison between the PMEmo performances.
\\ \indent
Finally Chapter 7, draws the conclusions and outlines possible future research directions.

\section{Application fields}
\indent
The work proposed in this thesis finds potential application in several fields. Thanks to the work of PMEmo that created a large dataset containing emotion annotations and electrodermal activity signal, we have the possibility to study the relationship between music emotion and physiological signals.
\\
Music Browsing can be an important field of application, because it helps in general in finding, generally in large datasets, what music user are looking for. For example one application could be to create a playlist based on the emotion that songs produce in each of us.
Another important application is given by understanding the relationship between music and emotion, which is a well known relationship but hard to find structural connection between the two.




